%-----------------------------------------------------------
% LaTeX template for University of Warwick exams
%-----------------------------------------------------------


\usepackage{fancyeq}
\usepackage{multicol}

%\usepackage[nomap]{FiraMono}

% configure the heading 
\ModuleCode{CS141}
\ExamCode{CS1419\_A}
\ModuleName{Functional Programming}
\ExamPeriod{September 2021}
\ExamsName{First Year Examinations}
\TimeAllowed{2 hours}
\QuestionInstructions{There are \textbf{SIX} questions. Candidates should attempt \textbf{FOUR} questions.}
\OtherInstructions{Instructions specific to this module:
\begin{itemize}
    \item The questions are not in order of difficulty.
    \item Unless stated otherwise, you should assume that library functions are defined as shown in the module guide.
\end{itemize}
General exam instructions follow on the next page.}

\setminted[text]{fontsize=\small}
\setminted[haskell]{fontsize=\small}
\setminted[bash]{fontsize=\small}
\newcommand{\haskellTopIn}[1]{\mintinline[fontsize=\small]{text}{#1}}
\newcommand{\haskellIn}[1]{\mintinline[fontsize=\small,breaklines]{text}{#1}}
\newcommand{\bashIn}[1]{\mintinline[fontsize=\small]{bash}{#1}}

\usepackage[nomap]{FiraMono}

\usepackage{amsmath}
\usepackage{microtype}
\DisableLigatures[f]{encoding = *, family = tt* }

\begin{document}
	\MakeHeading
	
	\begin{questions}
		%%% Question 1 - - - - - - - - - - - - - - - - - - - - - - - - - - - - - -
\question This question is about functional programming as a programming paradigm.

\makeatletter
\renewcommand{\verbatim@font}{\ttfamily \small}
\makeatother

\begin{parts}
    
    \part Reduce all of the following Haskell expressions to normal forms. Your answers \emph{must} include all reduction steps. No marks are awarded for answers which just state the normal form.
    
    \begin{subparts}
        
        \subpart[2] \haskellIn{last "jedi"}  \droppoints
        
        \begin{solution}
            \emph{Comprehension/Application.} For all parts of this question: 2 marks for a full, complete trace. 1 mark if there is one \emph{minor} error. 0 marks otherwise.
\begin{verbatim}
   last "jedi"
=> last "edi"
=> last "di"
=> last "i"
=> 'i'
\end{verbatim}
        \end{solution}
        
        \subpart[2] \haskellIn{reverse [reverse, (++) []]} \droppoints 
        
        \begin{solution}
        \emph{Comprehension.}
\begin{verbatim}
   reverse [reverse, (++) []]
=> reverse [(++) []] ++ [reverse]
=> reverse [] ++ [(++) []] ++ [reverse]
=> [] ++ [(++) []] ++ [reverse]
=> [(++) []] ++ [reverse]
=> [] ++ ((++) [] : [reverse])
=> (++) [] : [reverse]
== [(++) [], reverse]
\end{verbatim}
        \end{solution}
    
        \subpart[2] \haskellIn{fmap (fmap (\f a -> (*2) . f a)) (Just [(+),(-)])}  \droppoints
        
        \begin{solution}
        \emph{Comprehension.}
\begin{verbatim}
   fmap (fmap (\f a -> (*2) . f a)) (Just [(+),(-)])
=> Just (fmap (\f a -> (*2) . f a) [(+),(-)])
=> Just ((\f a -> (*2) . f a) (+) : 
   fmap (\f a -> (*2) . f a) [(-)])
=> Just ((\a -> (*2) . (+) a) : 
   fmap (\f a -> (*2) . f a) [(-)])
=> Just ((\a -> (*2) . (+) a) : 
   (\f a -> (*2) . f a) (-) : 
   fmap (\f a -> (*2) . f a) [])
=> Just ((\a -> (*2) . (+) a) : 
   (\a -> (*2) . (-) a) : 
   fmap (\f a -> (*2) . f a) [])
=> Just ((\a -> (*2) . (+) a) : 
   (\a -> (*2) . (-) a) : [])
== Just [\a -> (*2) . (+) a, \a -> (*2) . (-) a]
\end{verbatim}
        \end{solution}
    
        \subpart[2] \haskellIn{length (take 2 (repeat repeat))} \droppoints
        
        \begin{solution}
            \emph{Comprehension.}
\begin{verbatim}
   length (take 2 (repeat repeat))
=> length (take 2 (repeat : repeat repeat))
=> length (repeat : take 1 (repeat repeat))
=> 1 + length (take 1 (repeat repeat))
=> 1 + length (take 1 (repeat : repeat repeat))
=> 1 + length (repeat : take 0 (repeat repeat))
=> 1 + 1 + length (take 0 (repeat repeat))
=> 1 + 1 + length []
=> 1 + 1 + 0
=> 2
\end{verbatim}
        \end{solution}
    
        \subpart[2] \haskellIn{let a = Just 4 in pure (+) <*> a} \droppoints
        
        \begin{solution}
        \emph{Comprehension.} 
\begin{verbatim}
   let a = Just 4 in pure (+) <*> a
=> pure (+) <*> Just 4
=> Just (+) <*> Just 4
=> Just (4+)
\end{verbatim}
        \end{solution}
    \end{subparts}

    %\ifprintanswers \else \pagebreak \fi

    \part[5] Consider the following definition of \haskellIn{map}:
    \begin{small}
\begin{minted}{haskell}
map :: (a -> b) -> [a] -> [b]
map f []     = []
map f (x:xs) = f x : map f xs
\end{minted}
    \end{small}
    \haskellIn{map} above is defined in terms of explicit recursion. Define an equivalent function \haskellIn{map'} in terms of a single call to \haskellIn{foldl} with suitable arguments so that it is not explicitly recursive. \droppoints 
    
    \begin{solution}
    \emph{Application.} One possible answer is:
\begin{minted}{haskell}
map' f = foldl (\r x -> r ++ [f x]) []
\end{minted}
    \end{solution}
\part[5] Define a closed type family 
\begin{minted}{haskell}
Odd :: Nat -> Bool
\end{minted}
which determines whether a type-level natural number is odd or not. For example, \haskellIn{Odd ('Succ 'Zero)} should evaluate to \haskellIn{'True}. You may assume that the \haskellIn{Bool} type has been promoted to a kind. If you require any helper type families in order to define \haskellIn{Odd}, you should define those as well. \droppoints
\begin{solution}
    \emph{Application.} 2 marks for \haskellIn{Not} (or equivalent) and 3 marks for \haskellIn{Odd}.
\begin{minted}{haskell}
type family Not (b :: Bool) :: Bool where
  Not 'True = 'False
  Not 'False = 'True

type family Odd (n :: Nat) :: Bool where
  Odd 'Zero = 'False
  Odd ('Succ k) = Not (Odd k)
\end{minted}
\end{solution}

\part[5] A university student wants to write a Haskell program to manage her marks. For this purpose, every module is either incomplete and has no mark or is complete and has a mark of type \haskellIn{Int}. Define a suitable GADT to represent modules so that their type indicates whether they have been completed or not. Define any other types you may need to help you. \droppoints

\begin{solution}
    \emph{Comprehension/Application.}
\begin{minted}{haskell}
data Status = Complete | Incomplete
data Module (s :: Status) where
    Module :: Module 'Incomplete
    ModuleWithMark :: Int -> Module 'Complete
\end{minted}
\end{solution}

\end{parts}
 % done
		%%% Question 2 - - - - - - - - - - - - - - - - - - - - - - - - - - - - - -
\question This question is about recursive and higher-order functions. 

\begin{parts}

    \part[5] Write a function
\begin{minted}{haskell}
isBalanced :: String -> Bool
\end{minted}
    which checks whether a string only contains matching pairs of parentheses. As examples, \haskellIn{isBalanced "a((b)c)(d)"} should evaluate to \haskellIn{True} while \linebreak \haskellIn{isBalanced ":)"} and \haskellIn{isBalanced ")()("} should evaluate to \haskellIn{False}. \droppoints
    
    \begin{solution}
        \emph{Application}.
\begin{minted}{haskell}
isBalanced :: String -> Bool
isBalanced = check 0
    where check n "" = n == 0
          check n ('(':xs) = check (n+1) xs
          check n (')':xs)
           | n > 0 = check (n-1) xs
           | otherwise = False
          check n (_:xs) = check n xs
\end{minted}
    \end{solution}

    \part[4] Regardless of your implementation of \haskellIn{isBalanced}, briefly discuss what the time complexity of the function should be -- how many evaluation steps does it take relative to the size of the input \haskellIn{String}. For any given input, what is the worst case, what is the best case, and why? \droppoints

    \begin{solution}
        \emph{Comprehension/Application.} The function is $\mathcal{O}(n)$  with respect to the size of its input in the worst case (1 mark) since we must inspect every character in order (1 mark). The base case is constant \emph{regardless} of the size of the input (1 mark) since we can stop if we encounter a \haskellIn{')'} without first encountering a corresponding \haskellIn{'('} such as in \haskellIn{isBalanced ")averylongstringhere"} (1 mark).
    \end{solution}

    \part[5] Define a new function, \haskellIn{isBalanced'}, which does not hardcode the \haskellIn{'('} and \haskellIn{')'} characters and instead accepts arbitrary predicates of type \haskellIn{Char -> Bool} as arguments to determine whether something is an opening character or a closing character. 
    
    For example, once you have completed \haskellIn{isBalanced'}, it should then be possible to implement \haskellIn{isBalanced} as \haskellIn{isBalanced' (=='(') (==')')}. You may assume that the two predicates given to your function do not both return \haskellIn{True} for the same input. \droppoints

    \begin{solution}
        \emph{Application}.
\begin{minted}{haskell}
isBalanced' 
  :: (Char -> Bool) -> (Char -> Bool) -> String -> Bool
isBalanced' op cl = check 0
    where check n "" = n == 0
          check n (x:xs)
           | op x = check (n+1) xs
           | cl x && n > 0 = check (n-1) xs
           | cl x && n == 0 = False
           | otherwise = check n xs
\end{minted}
    \end{solution}    

    \part[5] In addition to deciding whether a string contains only matching opening and closing characters, we now also want to return the number of such matching pairs. Define a function \haskellIn{isBalanced''} whose type is the same as that of \haskellIn{isBalanced'} except that the return type is now \haskellIn{(Int, Bool)}. 
    
    For example, \haskellIn{isBalanced'' (=='[') (==']') "[[1]][0][]"} should evaluate to \haskellIn{(4, True)} while \haskellIn{isBalanced'' (=='[') (==']') "[[1]][0]]"} \linebreak should evaluate to \haskellIn{(3, False)}. In the case of an unmatched closing character, you should return the number of matching pairs found up until that point only. \droppoints

    \begin{solution}
        \emph{Application}.
\begin{minted}{haskell}
isBalanced'' 
  :: (Char -> Bool) -> (Char -> Bool) 
  -> String -> (Int, Bool)
isBalanced'' op cl = check 0 0
    where check n m "" = (m, n == 0)
          check n m (x:xs)
           | op x = check (n+1) m xs
           | cl x && n > 0 = check (n-1) (m+1) xs
           | cl x && n == 0 = (m, False)
           | otherwise = check n m xs
\end{minted}
    \end{solution} 

    \part[6] A very observant student notices that computational effects could greatly generalise the function. Define a new function \haskellIn{isBalanced'''} where you change the types of the predicates from \haskellIn{Char -> Bool} to \haskellIn{Char -> m Bool}, add an extra parameter of type \haskellIn{m ()} which will be called when a matching pair is found, and change the return type of the overall function to \haskellIn{m Bool} where \haskellIn{m} would be constrained by \haskellIn{Monad}. For example, the ``counting'' behaviour of \haskellIn{isBalanced''} could then be implemented using \haskellIn{isBalanced'''} and the state monad as follows:
    \begin{minted}{haskell}
isBalanced'' op cl xs = 
  runState (isBalanced''' op' cl' f xs) 0
  where op' = pure . op
        cl' = pure . cl
        f = get >>= \n -> put (n+1)
    \end{minted}
    Implement \haskellIn{isBalanced'''}. \droppoints

    \begin{solution}
        \emph{Application}.
\begin{minted}{haskell}
isBalanced''' 
  :: Monad m 
  => (Char -> m Bool) 
  -> (Char -> m Bool) 
  -> m ()
  -> String -> m Bool
isBalanced''' op cl f = check 0
    where check n "" = pure (n == 0)
          check n (x:xs) = do
            isOp <- op x
            if isOp then check (n+1) xs
            else do
              isCl <- cl x
              if isCl && n > 0 then f >> check (n-1) xs
              else if isCl && n == 0 then pure False
                   else check n xs
\end{minted}
    \end{solution}
\end{parts}
 % done
		%%% Question 3 - - - - - - - - - - - - - - - - - - - - - - - - - - - - - -
\question This question is about user-defined types and type classes.
\begin{parts}
    \part Consider the usual definition of length-indexed vectors in Haskell:
\begin{minted}{haskell}
data Vector (n :: Nat) a where 
  Empty :: Vector 'Zero a 
  Cons :: a -> Vector n a -> Vector ('Succ n) a
\end{minted}

    \begin{subparts}
        \subpart[3] The \haskellIn{Vector} type is a functor. Define a suitable instance of the \haskellIn{Functor} type class for it. \droppoints

        \begin{solution}
            \emph{Application.}
\begin{minted}{haskell}
instance Functor (Vector n) where
    fmap f Empty = Empty
    fmap f (Cons x xs) = Cons (f x) (fmap f xs)
\end{minted}
        \end{solution}

        \subpart[10] Prove that your instance of the \haskellIn{Functor} type class for the \haskellIn{Vector} type obeys both functor laws. \droppoints

        \begin{solution}
            \emph{Application.} 1 mark for identity base case, 3 marks for identity inductive step, 2 marks for fusion base case, 4 marks for fusion inductive step.
\begin{verbatim}
Identity:

  fmap id Empty
= Empty
= id Empty

  fmap id (Cons x xs)
= Cons (id x) (fmap id xs)
= Cons x (fmap id xs)
= Cons x (id xs)
= Cons x xs
= id (Cons x xs)

Fusion:

  fmap f (fmap g Empty)
= fmap f Empty
= Empty
= fmap (f.g) Empty

  fmap f (fmap g (Cons x xs))
= fmap f (Cons (g x) (fmap g xs))
= Cons (f (g x)) (fmap f (fmap g xs))
= Cons (f (g x)) (fmap (f.g) xs)
= Cons ((f.g) x) (fmap (f.g) xs)
= fmap (f.g) (Cons x xs)
\end{verbatim}
        \end{solution}
    \end{subparts}

\part Consider the following definition of a length-indexed zipper type for vectors where \haskellIn{n} represents the number of elements on the left of the view and \haskellIn{m} represents the number of elements on the right of the view:
\begin{minted}{haskell}
data IxZipper (n :: Nat) (m :: Nat) a where
  MkZipper :: Vector n a -> a -> Vector m a -> IxZipper n m a
\end{minted}
\begin{subparts}
    \subpart[5] For ordinary zippers, we have a function
    \vspace*{0.2cm}
\begin{minted}{haskell}
fromList :: [a] -> Zipper a
\end{minted}
    \vspace*{0.2cm}
    which constructs a zipper from a list, using the head of the list as the element in the view and all other elements as those right of the view. The function crashes for empty lists. Define an equivalent function named \haskellIn{fromVector} for the \haskellIn{IxZipper} type which takes a vector as the input and cannot be used on an empty vector. Additionally, explain how your solution accomplishes this. \droppoints

    \begin{solution}
        \emph{Application.}
\begin{minted}{haskell}
fromVector :: Vector ('Succ n) a -> IxZipper 'Zero n a
fromVector (Cons x xs) = MkZipper Empty x xs
\end{minted}
    By taking a \haskellIn{Vector} with a non-zero length, enforced by \haskellIn{'Succ n} in the explicit type signature, we ensure that the input is a vector with at least one element.
    \end{solution}

    \subpart[5] For ordinary zippers, we have functions
    \vspace*{0.2cm}
\begin{minted}{haskell}
leftShift :: Zipper a -> Zipper a 
rightShift :: Zipper a -> Zipper a
\end{minted}
    \vspace*{0.2cm}
    which shift the view to the left or right, respectively. If this is not possible, the input is returned unchanged. However, with our \haskellIn{IxZipper} type we can rule out that the functions may be called on zippers where there are no more elements to the left or right. Define two equivalent functions \haskellIn{leftShiftIx} and \haskellIn{rightShiftIx} so that they can only be called if the state of the \haskellIn{IxZipper} value will change. \droppoints

    \begin{solution}
        \emph{Application.}
\begin{minted}{haskell}
leftShift :: 
  IxZipper ('Succ n) m a -> IxZipper n ('Succ m) a
leftShift (MkZipper (Cons y ys) x r) = 
  MkZipper ys y (Cons x r)

rightShift :: 
  IxZipper n ('Succ m) a -> IxZipper ('Succ n) m a
rightShift (MkZipper l x (Cons y ys)) = 
  MkZipper (Cons x l) y ys
\end{minted}
    \end{solution}

    \subpart[2] The following property states that shifting a zipper's view to the right and then back to the left results in the same zipper we started with:
    \begin{center}
        \texttt{\small leftShift (rightShift (fromVector (Cons x (Cons y ys))))}\\
        \texttt{\small == fromVector (Cons x (Cons y ys)))}
    \end{center}
    Prove that this property holds. \droppoints

    \begin{solution}
        \emph{Application.}
\begin{verbatim}
  leftShift (rightShift (fromVector (Cons x (Cons y ys))))
= leftShift (rightShift (MkZipper Empty x (Cons y ys)))
= leftShift (MkZipper (Cons x Empty) y ys)
= MkZipper Empty x (Cons y ys)
= fromVector (Cons x (Cons y ys)))
\end{verbatim}
    \end{solution}
\end{subparts}

\end{parts}
 % done
		\allowdisplaybreaks
%%% Question 4 - - - - - - - - - - - - - - - - - - - - - - - - - - - - - -
\question This question is about equational reasoning. 
\begin{parts}
    
    \part[5] The \haskellIn{map} function does not change the length of a list:
    \begin{center}
        \texttt{\small length (map f xs) == length xs}
    \end{center}
    Prove that this property holds. \droppoints

    \begin{solution}
        \emph{Application.} 1 mark for the base case, 4 marks for the inductive case.
\begin{verbatim}
Proof by induction on xs.
Base case:
length (map f [])
= { map }
length []

Inductive case:
length (map f (x:xs))
= { map }
length (f x : map f xs)
= { length }
1 + length (map f xs)
= { IH }
1 + length xs
= { unapply length }
length (x:xs)
\end{verbatim}
    \end{solution}

    \part[20] Consider the following function which conses some value to the start of each list in a list of lists:
\begin{minted}{haskell}
prependAll :: a -> [[a]] -> [[a]]
prependAll x = map (x:)
\end{minted}
    We can use this to define another function which conses all values from a list to each list in a list of lists:
\begin{minted}{haskell}
prependProduct :: [a] -> [[a]] -> [[a]]
prependProduct xs yss = 
    concat (map (\x -> prependAll x yss) xs)
\end{minted}
    Since we cons every element from \haskellIn{xs} to every list in \haskellIn{yss}, the result of calling \haskellIn{prependProduct} is a list whose length is the product of the length of \haskellIn{xs} and the length of \haskellIn{yss}:
\begin{center}
    \texttt{\small length (prependProduct xs yss) == length xs * length yss}
\end{center}
    Prove this property with the help of the previous property you proved about \haskellIn{map}. Additionally, you may assume standard properties of arithmetic hold. If any other auxiliary lemmas are required, you should prove those. \droppoints

    \begin{solution}
        \emph{Application.}
\begin{verbatim}
Induction on xs.
Base case (main proof - 5 marks):
length (prependProduct [] yss)
= { applying prependProduct }
length (concat (map (\x -> prependAll x yss) []))
= { applying map }
length (concat [])
= { applying concat }
length []
= { applying length }
0
= { identity of multiplication }
0 * length yss
= { unapplying length }
length [] * length yss

For the inductive case, we need to prove that length 
distributes over ++:

length (xs ++ ys) == length xs + length ys

Note that the same proof was required for the summer paper, 
although the property was asked for explicitly there while 
it is implicitly required here:

Base case (lemma - 2 marks):
length ([] ++ ys) 
= { applying ++ }
  length ys 
= { left identity of + }
  0 + length ys
= { unapplying length }
  length [] + length ys

Inductive case (lemma - 5 marks):
  length ((x:xs) ++ ys)
= { applying ++ }
  length (x : (xs ++ ys))
= { applying length }
  1 + length (xs++ys)
= { induction hypothesis }
  1 + (length xs + length ys)
= { associativity of + }
  (1 + length xs) + length ys
= { unapplying length }
  length (x:xs) + length ys

Inductive case (main proof - 8 marks):
length (prependProduct (z:zs) yss)
= { applying prependProduct } 
length (concat (map (\x -> prependAll x yss) (z:zs)))
= { applying map }
length (concat ((prependAll z yss) : 
    map (\x -> prependAll x yss) zs))
= { applying concat }
length ((prependAll z yss) ++ 
    concat (map (\x -> prependAll x yss) zs))
= { distributivity of length over ++ } 
length (prependAll z yss) + 
    length (concat (map (\x -> prependAll x yss) zs))
= { applying prependAll }
length (map (z:) yss) + 
    length (concat (map (\x -> prependAll x yss) zs))
= { map doesn't change the length of a list }
length yss + 
    length (concat (map (\x -> prependAll x yss) zs))
= { induction hypothesis }
length yss + (length zs * length yss) 
= { arithmetic }
(1 + length zs) * length yss
= { unapplying length }
length (z:zs) * length yss 
\end{verbatim}
    \end{solution}
\end{parts}
 % done
		
%%% Question 5 - - - - - - - - - - - - - - - - - - - - - - - - - - - - - -
\question This question is about functors, applicative functors, and monads. You may assume that functor, applicative functor, and monad laws have been proved for all \haskellIn{Functor}, \haskellIn{Applicative}, and \haskellIn{Monad} instances except your own. You may \textbf{not} make use of GHC extensions such as \texttt{\small DeriveFunctor}. Consider the following type: % or maybe pairs?
\begin{minted}{haskell}
data U w u = W u | U (w (U w u))
\end{minted}
\begin{parts} 
    \part[5] Define a suitable instance of the \haskellIn{Functor} type class for the \haskellIn{U} type, adding suitable super-class constraints if necessary. \emph{Hint:} for your own benefit, remember to write down the specialised type signature of \haskellIn{fmap} and think about the types of the values you are working with. \droppoints 
    
    \begin{solution} \emph{Application.} 2 marks for the correct instance head, 1 mark for the base case, and 2 marks for the recursive case.
\begin{minted}{haskell}
instance Functor f => Functor (U f) where 
    fmap f (W u) = W (f u)
    fmap f (U wu) = U (fmap (fmap f) wu)
\end{minted}
    \end{solution}

    \part[10] Use induction to prove that your instance of the \haskellIn{Functor} type class for the \haskellIn{U} type obeys \emph{both} functor laws. \emph{Hint:} in the inductive cases, you may assume that your induction hypothesis is for \emph{all recursive occurrences} of the \haskellIn{U} type. \droppoints
    
    \begin{solution}
    Proofs (4 marks for identity, 6 marks for fusion):
\begin{verbatim}
Identity law. The proof is by induction. Base case (1 mark):

fmap id (W u) 
= { applying fmap }
W (id u)
= { applying id }
W u
= { unapplying id }
id (W u)

Inductive step (3 marks):

fmap id (U wu)
= { applying fmap }
U (fmap (fmap id) wu)
= { induction hypothesis holds for 
    all recursive occurrences }
U (fmap id wu)
= { type f is a functor: identity law }
U (id wu)
= { applying id }
U wu
= { unapplying id }
id (U wu)

Fusion law. The proof is by induction. Base case (2 marks):

fmap f (fmap g (W u))
= { applying fmap }
fmap f (W (g u))
= { applying fmap }
W (f (g u))
= { definition of function composition }
W ((f . g) u)
= { unapplying fmap }
fmap (f . g) (W u)

Inductive step (4 marks):

fmap f (fmap g (U wu))
= { applying fmap }
fmap f (U (fmap (fmap g) wu))
= { applying fmap }
U (fmap (fmap f) (fmap (fmap g) wu))
= { type f is a functor: fusion law }
U (fmap (fmap f . fmap g) wu)
= { induction hypothesis }
U (fmap (fmap (f . g)) wu)
= { unapplying fmap }
fmap (f . g) (U wu)
\end{verbatim}
    \end{solution}
    
    \part[10] Assume that there is a suitable instance of the \haskellIn{Applicative} type class for the \haskellIn{U} type. Now define a suitable instance of the \haskellIn{Monad} type class, including both \haskellIn{return} and \haskellIn{>>=}, for the \haskellIn{U} type, adding suitable super-class constraints if necessary. Your definition should obey the monad laws, although you do not need to prove this. \emph{Hint:} for your own benefit, remember to write down the specialised type signatures of \haskellIn{return} and \haskellIn{>>=} and think about the types of the values you are working with. \droppoints
    
    \begin{solution}
        \emph{Application.} 1 mark for \haskellIn{return} (if someone writes \haskellIn{pure} instead of \haskellIn{W}, award the mark as well), 2 marks for the \haskellIn{W} case for \haskellIn{>>=} and 4 marks for the \haskellIn{U} case. The remaining 3 marks are for the correct instance head, including identifying that a \haskellIn{Functor} constraint is sufficient for \haskellIn{f} -- 2 marks are available if a more restrictive constraint, such as \haskellIn{Monad} is used for \haskellIn{f}.
\begin{minted}{haskell}
instance Functor f => Monad (U f) where 
    return = W

    W u >>= g = g u
    U wu >>= g = U (fmap (>>= g) wu)
\end{minted}
    \end{solution}
\end{parts}
 % done
		%%% Question 6 - - - - - - - - - - - - - - - - - - - - - - - - - - - - - -
\question This question is about type-level programming. For this question, you may assume that all of GHC's language extensions are available to you. A library author wants to annotate configuration values at the type-level to reflect settings which have an impact on the library's security. The author currently has the following record type to represent configurations for the library which store a path to a file, represented as a \haskellIn{String}, and a value of type \haskellIn{Bool}:
\begin{minted}{haskell}
data Config = MkConfig {
    cfgPublicKeyPath :: String,
    cfgCheckExpired :: Bool
}
\end{minted}
\begin{parts}
    \part[4] The library author defines a new type to enumerate different security-related tags for the \haskellIn{Config} type based on modifications he is planning to the \haskellIn{Config} type later:
\begin{minted}{haskell}
data SecurityTag = NoPublicKey | NoCheckExpired
\end{minted}
    Show how to modify the \haskellIn{Config} type to be of kind \haskellIn{[SecurityTag] -> *} and explain which language extensions are needed and why. \droppoints   

    \begin{solution}
        \emph{Application.} \haskellIn{DataKinds} is needed to promote data types to kinds (1 mark) and \haskellIn{KindSignatures} is needed for the kind signature on \haskellIn{opts} (1 mark). 2 marks for the definition:
\begin{minted}{haskell}
data Config (opts :: [SecurityTag]) = MkConfig {
    cfgPublicKeyPath :: String,
    cfgCheckExpired :: Bool
}
\end{minted}
    \end{solution}

    \part[4] The library author now wants the type of \haskellIn{cfgPublicKeyPath} to vary depending on whether the list of \haskellIn{SecurityTag} types includes \haskellIn{NoPublicKey} or not. Define a closed type family
\begin{minted}{haskell}
IfFlag :: SecurityTag -> [SecurityTag] -> * -> * -> *
\end{minted}
    which checks whether the list of \haskellIn{SecurityTag} types includes the \linebreak \haskellIn{SecurityTag} type specified as the first argument. If so, then the first type of kind \haskellIn{*} should be returned. Otherwise, the second type of kind \haskellIn{*} should be returned. For example, \haskellIn{IfFlag 'NoPublicKey '[] Bool String} should evaluate to \haskellIn{String} and \haskellIn{IfFlag 'NoPublicKey '[NoPublicKey] Bool String} should evaluate to \haskellIn{Bool}.
    \droppoints 

    \begin{solution} \emph{Application.} 
\begin{minted}{haskell}
type family IfFlag 
    (x :: SecurityTag) 
    (xs :: [SecurityTag]) 
    (t :: *) (f :: *) :: * where

    IfFlag x '[] t f = f
    IfFlag x (x ': xs) t f = t
    IfFlag x (y ': xs) t f = IfFlag x xs t f
\end{minted}
    \end{solution}

    \part[2] With the help of \haskellIn{IfFlag}, modify your definition of the \haskellIn{Config} type so that \haskellIn{cfgPublicKeyPath} has type \haskellIn{String} if the list of \haskellIn{SecurityTag} types does not contain \haskellIn{NoPublicKey} or the unit type \haskellIn{()} otherwise. \droppoints

\begin{minted}{haskell}
data Config (opts :: [SecurityTag]) = MkConfig {
    cfgPublicKeyPath :: IfFlag 'NoPublicKey opts () String,
    cfgCheckExpired :: Bool
}
\end{minted}    

    \part[6] Define a singleton type variant of the \haskellIn{Bool} type, named \haskellIn{SBool} and explain which language extensions are needed and why. \droppoints

    \begin{solution}
        \emph{Comprehension.} In addition to the \haskellIn{DataKinds} extension to promote \haskellIn{Bool} to a kind (1 mark), \haskellIn{KindAnnotations} to annotate \haskellIn{b} with its kind (1 mark), we also require \haskellIn{GADTs} (1 mark) to customise the return types of the data constructors (1 mark). 2 marks for the definition (bookwork):
\begin{minted}{haskell}
data SBool (b :: Bool) where
    SFalse :: SBool 'False
    STrue :: SBool 'True
\end{minted}
    \end{solution}

    \part[4] With the help of \haskellIn{IfFlag}, modify your definition of the \haskellIn{Config} type so that \haskellIn{cfgCheckExpired} can only be set to \haskellIn{STrue} if the list of \haskellIn{SecurityTag} types contains \haskellIn{NoCheckExpired} or can only be set to \haskellIn{SFalse} otherwise. \droppoints

    \begin{solution}
        \emph{Application.}
\begin{minted}{haskell}
data Config (opts :: [SecurityTag]) = MkConfig {
  cfgPublicKeyPath :: 
    IfFlag 'NoPublicKey opts () String,
  cfgCheckExpired :: 
    IfFlag 'NoCheckExpired opts (SBool 'True) (SBool 'False)
}
\end{minted} 
    \end{solution}

    \part[5] Given a type-level list, we can construct a function type where the types contained in the type-level list represent the parameters. Define a closed type family \haskellIn{MkFun} of kind \haskellIn{[*] -> * -> *} so that, for example, \haskellIn{Fun (Cons Int (Cons String Empty)) Bool} evaluates to \haskellIn{Int -> String -> Bool}. That is, the first parameter is the list of parameter types, the second is the return type, and the result is the resulting function type. \droppoints 
    \begin{solution}
        \emph{Application.} 
\begin{minted}{haskell}
type family MkFun (f :: [*]) r where
    MkFun '[]      r = r
    MkFun (a ': f) r = a -> Fun f r
\end{minted}
    \end{solution}
\end{parts}
 % done
	\end{questions}
\end{document}
