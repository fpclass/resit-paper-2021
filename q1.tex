%%% Question 1 - - - - - - - - - - - - - - - - - - - - - - - - - - - - - -
\question This question is about functional programming as a programming paradigm.

\makeatletter
\renewcommand{\verbatim@font}{\ttfamily \small}
\makeatother

\begin{parts}
    
    \part Reduce all of the following Haskell expressions to normal forms. Your answers \emph{must} include all reduction steps. No marks are awarded for answers which just state the normal form.
    
    \begin{subparts}
        
        \subpart[2] \haskellIn{last "jedi"}  \droppoints
        
        \begin{solution}
            \emph{Comprehension/Application.} For all parts of this question: 2 marks for a full, complete trace. 1 mark if there is one \emph{minor} error. 0 marks otherwise.
\begin{verbatim}
   last "jedi"
=> last "edi"
=> last "di"
=> last "i"
=> 'i'
\end{verbatim}
        \end{solution}
        
        \subpart[2] \haskellIn{reverse [reverse, (++) []]} \droppoints 
        
        \begin{solution}
        \emph{Comprehension.}
\begin{verbatim}
   reverse [reverse, (++) []]
=> reverse [(++) []] ++ [reverse]
=> reverse [] ++ [(++) []] ++ [reverse]
=> [] ++ [(++) []] ++ [reverse]
=> [(++) []] ++ [reverse]
=> [] ++ ((++) [] : [reverse])
=> (++) [] : [reverse]
== [(++) [], reverse]
\end{verbatim}
        \end{solution}
    
        \subpart[2] \haskellIn{fmap (fmap (\f a -> (*2) . f a)) (Just [(+),(-)])}  \droppoints
        
        \begin{solution}
        \emph{Comprehension.}
\begin{verbatim}
   fmap (fmap (\f a -> (*2) . f a)) (Just [(+),(-)])
=> Just (fmap (\f a -> (*2) . f a) [(+),(-)])
=> Just ((\f a -> (*2) . f a) (+) : 
   fmap (\f a -> (*2) . f a) [(-)])
=> Just ((\a -> (*2) . (+) a) : 
   fmap (\f a -> (*2) . f a) [(-)])
=> Just ((\a -> (*2) . (+) a) : 
   (\f a -> (*2) . f a) (-) : 
   fmap (\f a -> (*2) . f a) [])
=> Just ((\a -> (*2) . (+) a) : 
   (\a -> (*2) . (-) a) : 
   fmap (\f a -> (*2) . f a) [])
=> Just ((\a -> (*2) . (+) a) : 
   (\a -> (*2) . (-) a) : [])
== Just [\a -> (*2) . (+) a, \a -> (*2) . (-) a]
\end{verbatim}
        \end{solution}
    
        \subpart[2] \haskellIn{length (take 2 (repeat repeat))} \droppoints
        
        \begin{solution}
            \emph{Comprehension.}
\begin{verbatim}
   length (take 2 (repeat repeat))
=> length (take 2 (repeat : repeat repeat))
=> length (repeat : take 1 (repeat repeat))
=> 1 + length (take 1 (repeat repeat))
=> 1 + length (take 1 (repeat : repeat repeat))
=> 1 + length (repeat : take 0 (repeat repeat))
=> 1 + 1 + length (take 0 (repeat repeat))
=> 1 + 1 + length []
=> 1 + 1 + 0
=> 2
\end{verbatim}
        \end{solution}
    
        \subpart[2] \haskellIn{let a = Just 4 in pure (+) <*> a} \droppoints
        
        \begin{solution}
        \emph{Comprehension.} 
\begin{verbatim}
   let a = Just 4 in pure (+) <*> a
=> pure (+) <*> Just 4
=> Just (+) <*> Just 4
=> Just (4+)
\end{verbatim}
        \end{solution}
    \end{subparts}

    %\ifprintanswers \else \pagebreak \fi

    \part[5] Consider the following definition of \haskellIn{map}:
    \begin{small}
\begin{minted}{haskell}
map :: (a -> b) -> [a] -> [b]
map f []     = []
map f (x:xs) = f x : map f xs
\end{minted}
    \end{small}
    \haskellIn{map} above is defined in terms of explicit recursion. Define an equivalent function \haskellIn{map'} in terms of a single call to \haskellIn{foldl} with suitable arguments so that it is not explicitly recursive. \droppoints 
    
    \begin{solution}
    \emph{Application.} One possible answer is:
\begin{minted}{haskell}
map' f = foldl (\r x -> r ++ [f x]) []
\end{minted}
    \end{solution}
\part[5] Define a closed type family 
\begin{minted}{haskell}
Odd :: Nat -> Bool
\end{minted}
which determines whether a type-level natural number is odd or not. For example, \haskellIn{Odd ('Succ 'Zero)} should evaluate to \haskellIn{'True}. You may assume that the \haskellIn{Bool} type has been promoted to a kind. If you require any helper type families in order to define \haskellIn{Odd}, you should define those as well. \droppoints
\begin{solution}
    \emph{Application.} 2 marks for \haskellIn{Not} (or equivalent) and 3 marks for \haskellIn{Odd}.
\begin{minted}{haskell}
type family Not (b :: Bool) :: Bool where
  Not 'True = 'False
  Not 'False = 'True

type family Odd (n :: Nat) :: Bool where
  Odd 'Zero = 'False
  Odd ('Succ k) = Not (Odd k)
\end{minted}
\end{solution}

\part[5] A university student wants to write a Haskell program to manage her marks. For this purpose, every module is either incomplete and has no mark or is complete and has a mark of type \haskellIn{Int}. Define a suitable GADT to represent modules so that their type indicates whether they have been completed or not. Define any other types you may need to help you. \droppoints

\begin{solution}
    \emph{Comprehension/Application.}
\begin{minted}{haskell}
data Status = Complete | Incomplete
data Module (s :: Status) where
    Module :: Module 'Incomplete
    ModuleWithMark :: Int -> Module 'Complete
\end{minted}
\end{solution}

\end{parts}
