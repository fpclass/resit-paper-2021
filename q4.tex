\allowdisplaybreaks
%%% Question 4 - - - - - - - - - - - - - - - - - - - - - - - - - - - - - -
\question This question is about equational reasoning. 
\begin{parts}
    
    \part[5] The \haskellIn{map} function does not change the length of a list:
    \begin{center}
        \texttt{\small length (map f xs) == length xs}
    \end{center}
    Prove that this property holds. \droppoints

    \begin{solution}
        \emph{Application.} 1 mark for the base case, 4 marks for the inductive case.
\begin{verbatim}
Proof by induction on xs.
Base case:
length (map f [])
= { map }
length []

Inductive case:
length (map f (x:xs))
= { map }
length (f x : map f xs)
= { length }
1 + length (map f xs)
= { IH }
1 + length xs
= { unapply length }
length (x:xs)
\end{verbatim}
    \end{solution}

    \part[20] Consider the following function which conses some value to the start of each list in a list of lists:
\begin{minted}{haskell}
prependAll :: a -> [[a]] -> [[a]]
prependAll x = map (x:)
\end{minted}
    We can use this to define another function which conses all values from a list to each list in a list of lists:
\begin{minted}{haskell}
prependProduct :: [a] -> [[a]] -> [[a]]
prependProduct xs yss = 
    concat (map (\x -> prependAll x yss) xs)
\end{minted}
    Since we cons every element from \haskellIn{xs} to every list in \haskellIn{yss}, the result of calling \haskellIn{prependProduct} is a list whose length is the product of the length of \haskellIn{xs} and the length of \haskellIn{yss}:
\begin{center}
    \texttt{\small length (prependProduct xs yss) == length xs * length yss}
\end{center}
    Prove this property with the help of the previous property you proved about \haskellIn{map}. Additionally, you may assume standard properties of arithmetic hold. If any other auxiliary lemmas are required, you should prove those. \droppoints

    \begin{solution}
        \emph{Application.}
\begin{verbatim}
Induction on xs.
Base case (main proof - 5 marks):
length (prependProduct [] yss)
= { applying prependProduct }
length (concat (map (\x -> prependAll x yss) []))
= { applying map }
length (concat [])
= { applying concat }
length []
= { applying length }
0
= { identity of multiplication }
0 * length yss
= { unapplying length }
length [] * length yss

For the inductive case, we need to prove that length 
distributes over ++:

length (xs ++ ys) == length xs + length ys

Note that the same proof was required for the summer paper, 
although the property was asked for explicitly there while 
it is implicitly required here:

Base case (lemma - 2 marks):
length ([] ++ ys) 
= { applying ++ }
  length ys 
= { left identity of + }
  0 + length ys
= { unapplying length }
  length [] + length ys

Inductive case (lemma - 5 marks):
  length ((x:xs) ++ ys)
= { applying ++ }
  length (x : (xs ++ ys))
= { applying length }
  1 + length (xs++ys)
= { induction hypothesis }
  1 + (length xs + length ys)
= { associativity of + }
  (1 + length xs) + length ys
= { unapplying length }
  length (x:xs) + length ys

Inductive case (main proof - 8 marks):
length (prependProduct (z:zs) yss)
= { applying prependProduct } 
length (concat (map (\x -> prependAll x yss) (z:zs)))
= { applying map }
length (concat ((prependAll z yss) : 
    map (\x -> prependAll x yss) zs))
= { applying concat }
length ((prependAll z yss) ++ 
    concat (map (\x -> prependAll x yss) zs))
= { distributivity of length over ++ } 
length (prependAll z yss) + 
    length (concat (map (\x -> prependAll x yss) zs))
= { applying prependAll }
length (map (z:) yss) + 
    length (concat (map (\x -> prependAll x yss) zs))
= { map doesn't change the length of a list }
length yss + 
    length (concat (map (\x -> prependAll x yss) zs))
= { induction hypothesis }
length yss + (length zs * length yss) 
= { arithmetic }
(1 + length zs) * length yss
= { unapplying length }
length (z:zs) * length yss 
\end{verbatim}
    \end{solution}
\end{parts}
